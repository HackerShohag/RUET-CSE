\documentclass{article}
\usepackage{lipsum}
\usepackage{tabularx}
\usepackage{multirow}
\usepackage{multicol}
\usepackage{program}
\usepackage{algorithm2e}
\usepackage{listings}
\usepackage{lipsum}
\usepackage{tabularx}
\usepackage{multirow}
\usepackage{multicol}
\usepackage{program}
\usepackage{algorithm2e}
\usepackage{listings}
\usepackage{xcolor}


% Define the style for C++ code
\lstdefinestyle{cppstyle}{
  language=C++,
  basicstyle=\ttfamily\small,
  keywordstyle=\color{blue},
  commentstyle=\color{green!40!black},
  stringstyle=\color{orange},
  numbers=left,
  numberstyle=\tiny,
  stepnumber=1,
  numbersep=5pt,
  backgroundcolor=\color{gray!5},
  frame=single,
  rulecolor=\color{black!30},
  breaklines=true,
  breakatwhitespace=true,
  tabsize=4
}


\begin{document}

\lipsum[1]

\begin{table}[!hbt]
    \centering
    \caption{Obtained Marks.}
    \label{tab-marks}

    \begin{tabular}{|l|c|c|c|c|}
        \hline Name   & Math & Phy & Chem & English \\
        \hline Sakib  & 40   & 42  & 78   & 67      \\
        \hline Tonmoy & 40   & 42  & 78   & 67      \\
        \hline Shohag & 40   & 42  & 78   & 67      \\
        \hline
    \end{tabular}

\end{table}

\lipsum[2]

\begin{table}[!hbt]
    \centering
    \caption{Obtained Marks.}
    \label{tab-marks-with-tabularx}

    \begin{tabularx}{0.8\linewidth} {|X|X|X|X|X|}
        \hline
        \multirow{2}{*}{Name} & \multicolumn{4}{c|}{Subjects}                        \\
        \cline{2-5}
                              & Math                          & Phy & Chem & English \\
        \hline
        Sakib                 & 40                            & 42  & 78   & 67      \\
        \hline
        Tonmoy                & 40                            & 42  & 78   & 67      \\
        \hline
        Shohag                & 40                            & 42  & 78   & 67      \\
        \hline
    \end{tabularx}

\end{table}
This explains in the table \ref{tab-marks}, \ref{tab-marks-with-tabularx}.

\vfil
Fourier Series:
\begin{equation}
    f(x) = A_0 + \sum_{n=1}^{\infty} A_n \cdot \cos{(\frac{n\pi x}{L})} + \sum_{n=1}^{\infty} B_n \cdot \sin{(\frac{n\pi x}{l})}
\end{equation}

\section*{C++ Code Example:}

\begin{lstlisting}[style=cppstyle, caption={Your C++ Code}, label={lst:cppcode}]
#include <iostream>

int main() {
    std::cout << "Hello, World!" << std::endl;
    return 0;
}
\end{lstlisting}

\end{document}